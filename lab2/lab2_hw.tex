% Options for packages loaded elsewhere
\PassOptionsToPackage{unicode}{hyperref}
\PassOptionsToPackage{hyphens}{url}
%
\documentclass[
]{article}
\usepackage{amsmath,amssymb}
\usepackage{iftex}
\ifPDFTeX
  \usepackage[T1]{fontenc}
  \usepackage[utf8]{inputenc}
  \usepackage{textcomp} % provide euro and other symbols
\else % if luatex or xetex
  \usepackage{unicode-math} % this also loads fontspec
  \defaultfontfeatures{Scale=MatchLowercase}
  \defaultfontfeatures[\rmfamily]{Ligatures=TeX,Scale=1}
\fi
\usepackage{lmodern}
\ifPDFTeX\else
  % xetex/luatex font selection
\fi
% Use upquote if available, for straight quotes in verbatim environments
\IfFileExists{upquote.sty}{\usepackage{upquote}}{}
\IfFileExists{microtype.sty}{% use microtype if available
  \usepackage[]{microtype}
  \UseMicrotypeSet[protrusion]{basicmath} % disable protrusion for tt fonts
}{}
\makeatletter
\@ifundefined{KOMAClassName}{% if non-KOMA class
  \IfFileExists{parskip.sty}{%
    \usepackage{parskip}
  }{% else
    \setlength{\parindent}{0pt}
    \setlength{\parskip}{6pt plus 2pt minus 1pt}}
}{% if KOMA class
  \KOMAoptions{parskip=half}}
\makeatother
\usepackage{xcolor}
\usepackage[margin=1in]{geometry}
\usepackage{color}
\usepackage{fancyvrb}
\newcommand{\VerbBar}{|}
\newcommand{\VERB}{\Verb[commandchars=\\\{\}]}
\DefineVerbatimEnvironment{Highlighting}{Verbatim}{commandchars=\\\{\}}
% Add ',fontsize=\small' for more characters per line
\usepackage{framed}
\definecolor{shadecolor}{RGB}{248,248,248}
\newenvironment{Shaded}{\begin{snugshade}}{\end{snugshade}}
\newcommand{\AlertTok}[1]{\textcolor[rgb]{0.94,0.16,0.16}{#1}}
\newcommand{\AnnotationTok}[1]{\textcolor[rgb]{0.56,0.35,0.01}{\textbf{\textit{#1}}}}
\newcommand{\AttributeTok}[1]{\textcolor[rgb]{0.13,0.29,0.53}{#1}}
\newcommand{\BaseNTok}[1]{\textcolor[rgb]{0.00,0.00,0.81}{#1}}
\newcommand{\BuiltInTok}[1]{#1}
\newcommand{\CharTok}[1]{\textcolor[rgb]{0.31,0.60,0.02}{#1}}
\newcommand{\CommentTok}[1]{\textcolor[rgb]{0.56,0.35,0.01}{\textit{#1}}}
\newcommand{\CommentVarTok}[1]{\textcolor[rgb]{0.56,0.35,0.01}{\textbf{\textit{#1}}}}
\newcommand{\ConstantTok}[1]{\textcolor[rgb]{0.56,0.35,0.01}{#1}}
\newcommand{\ControlFlowTok}[1]{\textcolor[rgb]{0.13,0.29,0.53}{\textbf{#1}}}
\newcommand{\DataTypeTok}[1]{\textcolor[rgb]{0.13,0.29,0.53}{#1}}
\newcommand{\DecValTok}[1]{\textcolor[rgb]{0.00,0.00,0.81}{#1}}
\newcommand{\DocumentationTok}[1]{\textcolor[rgb]{0.56,0.35,0.01}{\textbf{\textit{#1}}}}
\newcommand{\ErrorTok}[1]{\textcolor[rgb]{0.64,0.00,0.00}{\textbf{#1}}}
\newcommand{\ExtensionTok}[1]{#1}
\newcommand{\FloatTok}[1]{\textcolor[rgb]{0.00,0.00,0.81}{#1}}
\newcommand{\FunctionTok}[1]{\textcolor[rgb]{0.13,0.29,0.53}{\textbf{#1}}}
\newcommand{\ImportTok}[1]{#1}
\newcommand{\InformationTok}[1]{\textcolor[rgb]{0.56,0.35,0.01}{\textbf{\textit{#1}}}}
\newcommand{\KeywordTok}[1]{\textcolor[rgb]{0.13,0.29,0.53}{\textbf{#1}}}
\newcommand{\NormalTok}[1]{#1}
\newcommand{\OperatorTok}[1]{\textcolor[rgb]{0.81,0.36,0.00}{\textbf{#1}}}
\newcommand{\OtherTok}[1]{\textcolor[rgb]{0.56,0.35,0.01}{#1}}
\newcommand{\PreprocessorTok}[1]{\textcolor[rgb]{0.56,0.35,0.01}{\textit{#1}}}
\newcommand{\RegionMarkerTok}[1]{#1}
\newcommand{\SpecialCharTok}[1]{\textcolor[rgb]{0.81,0.36,0.00}{\textbf{#1}}}
\newcommand{\SpecialStringTok}[1]{\textcolor[rgb]{0.31,0.60,0.02}{#1}}
\newcommand{\StringTok}[1]{\textcolor[rgb]{0.31,0.60,0.02}{#1}}
\newcommand{\VariableTok}[1]{\textcolor[rgb]{0.00,0.00,0.00}{#1}}
\newcommand{\VerbatimStringTok}[1]{\textcolor[rgb]{0.31,0.60,0.02}{#1}}
\newcommand{\WarningTok}[1]{\textcolor[rgb]{0.56,0.35,0.01}{\textbf{\textit{#1}}}}
\usepackage{graphicx}
\makeatletter
\def\maxwidth{\ifdim\Gin@nat@width>\linewidth\linewidth\else\Gin@nat@width\fi}
\def\maxheight{\ifdim\Gin@nat@height>\textheight\textheight\else\Gin@nat@height\fi}
\makeatother
% Scale images if necessary, so that they will not overflow the page
% margins by default, and it is still possible to overwrite the defaults
% using explicit options in \includegraphics[width, height, ...]{}
\setkeys{Gin}{width=\maxwidth,height=\maxheight,keepaspectratio}
% Set default figure placement to htbp
\makeatletter
\def\fps@figure{htbp}
\makeatother
\setlength{\emergencystretch}{3em} % prevent overfull lines
\providecommand{\tightlist}{%
  \setlength{\itemsep}{0pt}\setlength{\parskip}{0pt}}
\setcounter{secnumdepth}{-\maxdimen} % remove section numbering
\ifLuaTeX
  \usepackage{selnolig}  % disable illegal ligatures
\fi
\IfFileExists{bookmark.sty}{\usepackage{bookmark}}{\usepackage{hyperref}}
\IfFileExists{xurl.sty}{\usepackage{xurl}}{} % add URL line breaks if available
\urlstyle{same}
\hypersetup{
  pdftitle={Lab 2 Homework},
  pdfauthor={Liana Williams},
  hidelinks,
  pdfcreator={LaTeX via pandoc}}

\title{Lab 2 Homework}
\author{Liana Williams}
\date{2024-01-16}

\begin{document}
\maketitle

\hypertarget{instructions}{%
\subsection{Instructions}\label{instructions}}

Answer the following questions and complete the exercises in RMarkdown.
Please embed all of your code and push your final work to your
repository. Your final lab report should be organized, clean, and run
free from errors. Remember, you must remove the \texttt{\#} for the
included code chunks to run. Be sure to add your name to the author
header above.

Make sure to use the formatting conventions of RMarkdown to make your
report neat and clean!

\begin{enumerate}
\def\labelenumi{\arabic{enumi}.}
\tightlist
\item
  What is a vector in R?
\end{enumerate}

\begin{Shaded}
\begin{Highlighting}[]
\CommentTok{\#A vector is a data structure in R.  }
\CommentTok{\#It is essentially multiple variables of the same type grouped together under a single piece of memory. }
\end{Highlighting}
\end{Shaded}

\begin{enumerate}
\def\labelenumi{\arabic{enumi}.}
\setcounter{enumi}{1}
\tightlist
\item
  What is a data matrix in R?
\end{enumerate}

\begin{Shaded}
\begin{Highlighting}[]
\CommentTok{\#Data matrices are a series of stacked vectors, similar to a data table.}
\end{Highlighting}
\end{Shaded}

\begin{enumerate}
\def\labelenumi{\arabic{enumi}.}
\setcounter{enumi}{2}
\tightlist
\item
  Below are data collected by three scientists (Jill, Steve, Susan in
  order) measuring temperatures of eight hot springs. Run this code
  chunk to create the vectors.
\end{enumerate}

\begin{Shaded}
\begin{Highlighting}[]
\NormalTok{spring\_1 }\OtherTok{\textless{}{-}} \FunctionTok{c}\NormalTok{(}\FloatTok{36.25}\NormalTok{, }\FloatTok{35.40}\NormalTok{, }\FloatTok{35.30}\NormalTok{)}
\NormalTok{spring\_2 }\OtherTok{\textless{}{-}} \FunctionTok{c}\NormalTok{(}\FloatTok{35.15}\NormalTok{, }\FloatTok{35.35}\NormalTok{, }\FloatTok{33.35}\NormalTok{)}
\NormalTok{spring\_3 }\OtherTok{\textless{}{-}} \FunctionTok{c}\NormalTok{(}\FloatTok{30.70}\NormalTok{, }\FloatTok{29.65}\NormalTok{, }\FloatTok{29.20}\NormalTok{)}
\NormalTok{spring\_4 }\OtherTok{\textless{}{-}} \FunctionTok{c}\NormalTok{(}\FloatTok{39.70}\NormalTok{, }\FloatTok{40.05}\NormalTok{, }\FloatTok{38.65}\NormalTok{)}
\NormalTok{spring\_5 }\OtherTok{\textless{}{-}} \FunctionTok{c}\NormalTok{(}\FloatTok{31.85}\NormalTok{, }\FloatTok{31.40}\NormalTok{, }\FloatTok{29.30}\NormalTok{)}
\NormalTok{spring\_6 }\OtherTok{\textless{}{-}} \FunctionTok{c}\NormalTok{(}\FloatTok{30.20}\NormalTok{, }\FloatTok{30.65}\NormalTok{, }\FloatTok{29.75}\NormalTok{)}
\NormalTok{spring\_7 }\OtherTok{\textless{}{-}} \FunctionTok{c}\NormalTok{(}\FloatTok{32.90}\NormalTok{, }\FloatTok{32.50}\NormalTok{, }\FloatTok{32.80}\NormalTok{)}
\NormalTok{spring\_8 }\OtherTok{\textless{}{-}} \FunctionTok{c}\NormalTok{(}\FloatTok{36.80}\NormalTok{, }\FloatTok{36.45}\NormalTok{, }\FloatTok{33.15}\NormalTok{)}
\end{Highlighting}
\end{Shaded}

\hypertarget{i-ran-the-code-above}{%
\subsection{\^{} I ran the code above! :)}\label{i-ran-the-code-above}}

\begin{enumerate}
\def\labelenumi{\arabic{enumi}.}
\setcounter{enumi}{3}
\tightlist
\item
  Build a data matrix that has the springs as rows and the columns as
  scientists.
\end{enumerate}

\begin{Shaded}
\begin{Highlighting}[]
\NormalTok{hotspring\_temp }\OtherTok{\textless{}{-}} \FunctionTok{c}\NormalTok{(spring\_1, spring\_2, spring\_3, spring\_4, spring\_5, spring\_6, spring\_7, spring\_8)}
\NormalTok{hotspring\_temp}
\end{Highlighting}
\end{Shaded}

\begin{verbatim}
##  [1] 36.25 35.40 35.30 35.15 35.35 33.35 30.70 29.65 29.20 39.70 40.05 38.65
## [13] 31.85 31.40 29.30 30.20 30.65 29.75 32.90 32.50 32.80 36.80 36.45 33.15
\end{verbatim}

\begin{Shaded}
\begin{Highlighting}[]
\CommentTok{\#Create a new object called \textquotesingle{}hotspring\_temp\textquotesingle{} using the \textasciigrave{}c\textasciigrave{} command to combine the vectors into one.}
\end{Highlighting}
\end{Shaded}

\begin{Shaded}
\begin{Highlighting}[]
\NormalTok{hotspring\_temp\_matrix }\OtherTok{\textless{}{-}} \FunctionTok{matrix}\NormalTok{(hotspring\_temp, }\AttributeTok{nrow =} \DecValTok{8}\NormalTok{, }\AttributeTok{byrow =}\NormalTok{ T)}
\NormalTok{hotspring\_temp\_matrix}
\end{Highlighting}
\end{Shaded}

\begin{verbatim}
##       [,1]  [,2]  [,3]
## [1,] 36.25 35.40 35.30
## [2,] 35.15 35.35 33.35
## [3,] 30.70 29.65 29.20
## [4,] 39.70 40.05 38.65
## [5,] 31.85 31.40 29.30
## [6,] 30.20 30.65 29.75
## [7,] 32.90 32.50 32.80
## [8,] 36.80 36.45 33.15
\end{verbatim}

\begin{Shaded}
\begin{Highlighting}[]
\CommentTok{\#Create a \textquotesingle{}hotspring\_temp\_matrix\textquotesingle{} using the \textquotesingle{}matrix()\textquotesingle{} command. Then tell R how to organize}
\CommentTok{\#\textquotesingle{}hotspring\_temp\textquotesingle{} vector organized by nrow and byrow commands. }
\end{Highlighting}
\end{Shaded}

\begin{enumerate}
\def\labelenumi{\arabic{enumi}.}
\setcounter{enumi}{4}
\tightlist
\item
  The names of the springs are 1.Bluebell Spring, 2.Opal Spring,
  3.Riverside Spring, 4.Too Hot Spring, 5.Mystery Spring, 6.Emerald
  Spring, 7.Black Spring, 8.Pearl Spring. Name the rows and columns in
  the data matrix. Start by making two new vectors with the names, then
  use \texttt{colnames()} and \texttt{rownames()} to name the columns
  and rows.
\end{enumerate}

\begin{Shaded}
\begin{Highlighting}[]
\NormalTok{scientists }\OtherTok{\textless{}{-}} \FunctionTok{c}\NormalTok{(}\StringTok{"Jill"}\NormalTok{, }\StringTok{"Steve"}\NormalTok{, }\StringTok{"Susan"}\NormalTok{)}
\NormalTok{scientists}
\end{Highlighting}
\end{Shaded}

\begin{verbatim}
## [1] "Jill"  "Steve" "Susan"
\end{verbatim}

\begin{Shaded}
\begin{Highlighting}[]
\CommentTok{\#scientists gets the column names of the three scientists }
\end{Highlighting}
\end{Shaded}

\begin{Shaded}
\begin{Highlighting}[]
\NormalTok{springs }\OtherTok{\textless{}{-}} \FunctionTok{c}\NormalTok{(}\StringTok{"Bluebell\_Spring"}\NormalTok{,}\StringTok{"Opal\_Spring"}\NormalTok{,}\StringTok{"Riverside\_Spring"}\NormalTok{,}\StringTok{"Too\_Hot\_Spring"}\NormalTok{,}\StringTok{"Mystery\_Spring"}\NormalTok{,}\StringTok{"Emerald\_Spring"}\NormalTok{,}\StringTok{"Black\_Spring"}\NormalTok{,}\StringTok{"Pearl\_Spring"}\NormalTok{)}
\NormalTok{springs}
\end{Highlighting}
\end{Shaded}

\begin{verbatim}
## [1] "Bluebell_Spring"  "Opal_Spring"      "Riverside_Spring" "Too_Hot_Spring"  
## [5] "Mystery_Spring"   "Emerald_Spring"   "Black_Spring"     "Pearl_Spring"
\end{verbatim}

\begin{Shaded}
\begin{Highlighting}[]
\CommentTok{\#springs gets the row names for each respective spring number}
\end{Highlighting}
\end{Shaded}

\begin{Shaded}
\begin{Highlighting}[]
\FunctionTok{colnames}\NormalTok{(hotspring\_temp\_matrix) }\OtherTok{\textless{}{-}}\NormalTok{ scientists}
\CommentTok{\#name the columns using \textquotesingle{}colnames()\textquotesingle{} with the vector scientists}
\end{Highlighting}
\end{Shaded}

\begin{Shaded}
\begin{Highlighting}[]
\FunctionTok{rownames}\NormalTok{(hotspring\_temp\_matrix) }\OtherTok{\textless{}{-}}\NormalTok{ springs}
\CommentTok{\#Name the rows using \textasciigrave{}rownames()\textasciigrave{} with the vector springs.}
\end{Highlighting}
\end{Shaded}

\begin{Shaded}
\begin{Highlighting}[]
\NormalTok{hotspring\_temp\_matrix}
\end{Highlighting}
\end{Shaded}

\begin{verbatim}
##                   Jill Steve Susan
## Bluebell_Spring  36.25 35.40 35.30
## Opal_Spring      35.15 35.35 33.35
## Riverside_Spring 30.70 29.65 29.20
## Too_Hot_Spring   39.70 40.05 38.65
## Mystery_Spring   31.85 31.40 29.30
## Emerald_Spring   30.20 30.65 29.75
## Black_Spring     32.90 32.50 32.80
## Pearl_Spring     36.80 36.45 33.15
\end{verbatim}

\begin{enumerate}
\def\labelenumi{\arabic{enumi}.}
\setcounter{enumi}{5}
\tightlist
\item
  Calculate the mean temperature of all eight springs.
\end{enumerate}

\begin{Shaded}
\begin{Highlighting}[]
\NormalTok{avg\_temp }\OtherTok{\textless{}{-}} \FunctionTok{rowMeans}\NormalTok{(hotspring\_temp\_matrix)}
\NormalTok{avg\_temp}
\end{Highlighting}
\end{Shaded}

\begin{verbatim}
##  Bluebell_Spring      Opal_Spring Riverside_Spring   Too_Hot_Spring 
##         35.65000         34.61667         29.85000         39.46667 
##   Mystery_Spring   Emerald_Spring     Black_Spring     Pearl_Spring 
##         30.85000         30.20000         32.73333         35.46667
\end{verbatim}

\begin{Shaded}
\begin{Highlighting}[]
\CommentTok{\#avg\_temp get the sum of all the rows in the hotspringmatrix; output is 8 springs each with the mean}
\end{Highlighting}
\end{Shaded}

\begin{enumerate}
\def\labelenumi{\arabic{enumi}.}
\setcounter{enumi}{6}
\tightlist
\item
  Add this as a new column in the data matrix.\\
  Add this information to the data matrix. Hint: you are adding a row,
  not a column.
\end{enumerate}

\begin{Shaded}
\begin{Highlighting}[]
\FunctionTok{cbind}\NormalTok{(hotspring\_temp\_matrix, avg\_temp)}
\end{Highlighting}
\end{Shaded}

\begin{verbatim}
##                   Jill Steve Susan avg_temp
## Bluebell_Spring  36.25 35.40 35.30 35.65000
## Opal_Spring      35.15 35.35 33.35 34.61667
## Riverside_Spring 30.70 29.65 29.20 29.85000
## Too_Hot_Spring   39.70 40.05 38.65 39.46667
## Mystery_Spring   31.85 31.40 29.30 30.85000
## Emerald_Spring   30.20 30.65 29.75 30.20000
## Black_Spring     32.90 32.50 32.80 32.73333
## Pearl_Spring     36.80 36.45 33.15 35.46667
\end{verbatim}

\begin{Shaded}
\begin{Highlighting}[]
\CommentTok{\#c is for column so therefore bind avg\_temp to the hotspringmatrix}
\end{Highlighting}
\end{Shaded}

\begin{enumerate}
\def\labelenumi{\arabic{enumi}.}
\setcounter{enumi}{7}
\tightlist
\item
  Show Susan's value for Opal Spring only.
\end{enumerate}

\begin{Shaded}
\begin{Highlighting}[]
\NormalTok{hotspring\_temp\_matrix[}\DecValTok{2}\NormalTok{,}\DecValTok{3}\NormalTok{]}
\end{Highlighting}
\end{Shaded}

\begin{verbatim}
## [1] 33.35
\end{verbatim}

\begin{Shaded}
\begin{Highlighting}[]
\CommentTok{\#the first number is the column and the second number is the row}
\end{Highlighting}
\end{Shaded}

\begin{enumerate}
\def\labelenumi{\arabic{enumi}.}
\setcounter{enumi}{8}
\tightlist
\item
  Calculate the mean for Jill's column only.
\end{enumerate}

\begin{Shaded}
\begin{Highlighting}[]
\NormalTok{Jill\_col }\OtherTok{\textless{}{-}}\NormalTok{ hotspring\_temp\_matrix[ ,}\DecValTok{1}\NormalTok{]}
\FunctionTok{mean}\NormalTok{(Jill\_col)}
\end{Highlighting}
\end{Shaded}

\begin{verbatim}
## [1] 34.19375
\end{verbatim}

\begin{Shaded}
\begin{Highlighting}[]
\CommentTok{\#calculate the mean of Jill\textquotesingle{}s column}
\end{Highlighting}
\end{Shaded}

\begin{enumerate}
\def\labelenumi{\arabic{enumi}.}
\setcounter{enumi}{9}
\tightlist
\item
  Use the data matrix to perform one calculation or operation of your
  interest.
\end{enumerate}

\begin{Shaded}
\begin{Highlighting}[]
\NormalTok{hotspring\_temp\_matrix[}\DecValTok{1}\SpecialCharTok{:}\DecValTok{8}\NormalTok{]}
\end{Highlighting}
\end{Shaded}

\begin{verbatim}
## [1] 36.25 35.15 30.70 39.70 31.85 30.20 32.90 36.80
\end{verbatim}

\begin{Shaded}
\begin{Highlighting}[]
\CommentTok{\#Use a colon \textasciigrave{}:\textasciigrave{} to selects the temperatures for each of the springs in Jill\textquotesingle{}s column. }
\end{Highlighting}
\end{Shaded}

\hypertarget{push-your-final-code-to-github}{%
\subsection{Push your final code to
GitHub!}\label{push-your-final-code-to-github}}

Please be sure that you check the \texttt{keep\ md} file in the knit
preferences.

\end{document}
